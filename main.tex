\documentclass[12pt]{article}
% \usepackage[utf8]{inputenc}
\usepackage[spanish]{babel}
\usepackage{csquotes}
\usepackage{titlesec}
\usepackage{titling}
\usepackage{fontspec}
\setmainfont{Times New Roman}
%\usepackage{cite}
%\usepackage{times}
\usepackage{parskip}
\usepackage{url}
\usepackage{setspace}
\usepackage{float}
%\usepackage[fixlanguage]{babelbib}
%\selectbiblanguage{spanish}
%\usepackage{todonotes}
\usepackage{graphicx}
\newcommand{\sectionbreak}{\clearpage}
%\renewcommand{\baselinestretch}{2.0}

    
\usepackage[
backend=biber,
style=ieee,
]{biblatex}
\addbibresource{mybib.bib} 

%\setlength{\parindent}{0 pt}
\titleformat{\chapter}[display]
  {\normalfont\bfseries}{}{0pt}{\Large}
\begin{document}

%\maketitle
\begin{titlepage}
        \vspace{-1.0cm}
       \includegraphics[width=0.6\textwidth]{images/LOGO.jpg}
   \begin{center}
       \vspace*{2cm}

       \textbf{TITULO DEL DOCUMENTO}

       \vspace{0.5cm}
       \vspace{4cm}
        por\\
       \textbf{Autor}

       \vfill
            Trabajo de Título presentado a la\\
Facultad de Ingeniería de la Universidad Católica de Temuco\\
Para Optar al Título de Ingeniero Civil Informático.
            
       %\vspace{0.8cm}
     
            
       \textbf{- Temuco, 2021 -}
      \vspace{-1cm}
            
   \end{center}
\end{titlepage}
\newpage
\tableofcontents
\newpage
\listoftables
\listoffigures
\begin{doublespace}
\doublespacing
\section{Resumen}
\section{Abstract}
\section{Introducción}
\section{Objetivos}
\subsection{Objetivo General}
%Implementar un sistema de intranet que soporte los procesos de gestión y comunicación de la empresa.
\subsection{Objetivos Específicos}

\section{Marco Teórico}

\section{Metodología}

\subsection{Herramientas y Tecnologías}

\subsection{Pruebas}

\section{Análisis y Requerimientos}

\section{Implementación}

\section{Resultados}

\section{Conclusiones}



\end{doublespace}
\printbibliography
\appendix
\section{Apéndice}
\input{apendix.tex}
\end{document}
